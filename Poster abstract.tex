\documentclass[12pt]{article}
\usepackage[utf8]{inputenc}
\usepackage{sigsam, amsmath}
\newtheorem{theorem}{Theorem}

\usepackage[ruled,vlined,linesnumbered]{algorithm2e}
\SetEndCharOfAlgoLine{}

% additional commands
\newcommand{\F}{\mathbb{F}}
\newcommand{\FF}{\F}
\renewcommand{\O}{\mathcal{O}}
\newcommand{\Ell}{\mathrm{Ell}}
\newcommand{\Cl}{\mathcal{C}}
\newcommand{\ms}{\mathrm{\ ms}}
\newcommand{\s}{\mathrm{\ s}}
\newcommand{\frakell}{\mathfrak{l}}
\newcommand{\Walk}{\textsc{Walk}}
\newcommand{\Step}{\textsc{Step}}

% leave as is
\issue{TBA}
\articlehead{TBA}
\titlehead{Isogeny-based cryptography in Julia/Nemo: a case study}
\authorhead{Jean Kieffer, Luca De Feo}
\setcounter{page}{1}

\begin{document}

\title{Isogeny-based cryptography in Julia/Nemo: a case study}

\author{Jean Kieffer \\
\'Ecole normale sup\'erieure de Paris \& INRIA, Universit\'e Paris-Saclay \\
\url{jean.kieffer@ens.fr}
\and
Luca De Feo \\
Universit\'e de Versailles \& INRIA, Universit\'e Paris-Saclay \\
\url{luca.defeo@polytechnique.edu}}

\maketitle

\begin{abstract}
The Couveignes--Rostovtsev--Stolbunov key-exchange protocol based on
    isogenies of elliptic curves is of interest because it may resist
    quantum attacks, but its efficient implementation remains a challenge.
    We briefly present the computations involved, and efficient
    algorithms to achieve the critical steps, with timing results for our implementations in Sage and Julia/Nemo.
\end{abstract}

\section{Introduction}
\label{sec:intro}

In \cite{RS}, Rostovtsev and Stolbunov proposed using isogenies between
ordinary elliptic curves over finite fields to build key-exchange and
public-key encryption schemes, rediscovering Couveignes' idea
\cite{Couveignes}. These schemes are based on the simply transitive action
of the abelian group $\Cl(\O)$, the class group of an order \(\O\) in a
quadratic field, on $\Ell_t(\O)$, the set of elliptic curves over a
(prime) finite field $\F_p$ with Frobenius trace $t$ and complex
multiplication by the order $\O$. Elements in $\Cl(\O)$ are represented
as products of ideals of prime norm, up to a certain bound. 
Each of these ideals \(\frakell\) is represented by its norm \(\ell\) and a
\emph{Frobenius eigenvalue} \(v_\ell\),
which is one of the two roots of \(X^2 - tX + p \pmod{\ell}\).
Computing the action of \(\Cl(\O)\) then amounts to following a path in
the so-called \emph{isogeny graph}, whose vertices are curves in $\Ell_t(\O)$ and edges are isogenies corresponding to ideals. 

The cryptosystem parameters are
a set \(L\) of prime ideals in \(\O\),
an exponent bound \(K\),
and a ``base point'' \(E_0\) in \(\Ell_t(\O)\).
Key exchange works as follows:
Alice samples a secret \(k_A\) from \([0,K]^L\)
and sends \(E_A = \Walk(E_0,k_A)\) to Bob;
Bob samples a secret \(k_B\) from \([0,K]^L\)
and sends \(E_B = \Walk(E_0,k_B)\) to Alice.
Their shared secret is \(E_{AB} \cong \Walk(E_A,k_B) \cong \Walk(E_B,k_A)\).
Each \(\Walk\) (Algorithm~\ref{alg:walk})
is a series of \(\Step\)s
for each of the various ideals \(\frakell\) in \(L\);
we describe two strategies for implementing \(\Step\) below,
comparing implementations in Sage \cite{Sage} and Nemo \cite{Nemo}. 

The choice of \(L\) is critical,
because the security of the scheme grows exponentially with \(\#L\)
(and only polynomially in \(K\)),
but the efficiency of each \(\Step\)
depends on the particular \(\frakell\) in~\(L\).
For our experiments,
we chose parameters aiming at the 128-bit security level
(comparable with contemporary cryptosystems such as AES)
against the best known classical attacks:
%To fix ideas, let us give an example of parameters achieving security comparable with contemporary cryptosystems, against the best known classical attacks. 
we take
$p~=~2^{500} + 55$, 
$K = 2$, 
$E_0: y^2 = x^3 + 5x^2 + x$,
and $L$ a set of 115 convenient primes with norms less than $1400$.
%(out of 222),
For each \(\frakell\) in \(L\),
the classical modular polynomial \(\Phi_\ell(X,Y)\)
is precomputed.

\begin{algorithm}
    \caption{{\sc Walk}: a walk in the isogeny graph}
    \label{alg:walk}
    \KwIn{An elliptic curve \(E\) in \(\mathrm{Ell}_t(\O)\)
    and a vector \(k = (k_\frakell)_{\frakell \in L} \in [0,K]^L\)
    }
    \KwOut{An elliptic curve \(E'\) such that \(E \to E'\)
    is an isogeny corresponding to \(\prod_{\frakell\in
    L}\frakell^{k_\frakell}\)}
    %
    \For{\(\frakell \in L\)}{
        \lFor{\(0 \le i < k_\frakell\)}{%
            \label{alg:walk:step-call}
            \(E \gets \textsc{Step}(E,\frakell)\)
        }
    }
    \Return{\(E\)}
\end{algorithm}


%Note that we cannot use all primes, but only about half of them: these primes are usually called \emph{Elkies primes} in point counting. One could try to find special curves, for which more primes could are Elkies. It is a major issue since the security of the scheme grows exponentially with the length of $L$ and only polynomially with $K$, and we return to this question later on.

\section{A first solution}

\begin{algorithm}
    \caption{{\sc Step}: a single step in the isogeny graph, using
    Bostan--Morain--Salvy--Schost}
    \label{alg:step}
    \KwIn{%
        $E\in \Ell_t(\O)$, 
        a prime ideal \(\frakell \leftrightarrow (\ell,v_\ell)\)
    }
    \KwOut{%
        $E'$ such that \(E \to E'\) is the edge in the isogeny graph
        corresponding to \(\frakell\)
    }
    %
    Evaluate \(F(X) = \Phi_\ell(X,j(E))\)
    \tcp*{\(j\) is the \(j\)-invariant; \(\Phi_\ell\) is precomputed}
    %
    \(\{j_1,j_2\} \gets \textsc{Roots}(F,\F_p)\);
    compute curves $E_1$, $E_2$ such that \(j_i = j(E_i)\)
    \label{alg:step:roots}
    \;
    %
    \(S_1 \gets \textsc{KernelOfIsogeny}(E,E_1,\ell)\)
    \label{alg:step:kernel}
    \;
    %
    \lIf{\((x^p \pmod{S_1}) = ([v_\ell]_*x \pmod{S_1})\)}{%
        \Return{\(E_1\)}
    }
    \lElse{%
        \Return{\(E_2\)}
    }
\end{algorithm}

Algorithm~\ref{alg:step}
is our first and most general implementation of \(\Step\).
In Line~\ref{alg:step:kernel},
we use the Bostan--Morain--Salvy--Schost algorithm \cite{BMSS}. 
This involves finding a rational function satisfying a differential
equation, which is solved using a Newton iteration,
before recovering 
the polynomials we want using the Berlekamp--Massey algorithm. 
The cost is dominated by finding the roots of $\Phi_\ell(X, j(E))$, 
which has degree $\ell + 1$, 
in Line~\ref{alg:step:roots}. 
For this, we use the Cantor--Zassenhaus algorithm, performing equal-degree factorisation only in degree 1.
The most costly step is computing $X^p\pmod{F}$.
We used Sage
(which relies on NTL \cite{NTL}, PARI \cite{Pari} and GMP \cite{GMP}
here), PARI
and Nemo built-ins for this common computation, 
giving the following results
(all timings are from a 3.20GHz Intel Xeon workstation).
\[
    \begin{array}{c|c|c|c|c|c|c|c|c|c|c|c}
        \deg F & 10 & 20 & 30 & 40 & 50 & 100 & 200 & 300 & 400 & 500 & 1000\\
        \hline
        \text{PARI}  & 13\ms & 43 \ms & 93 \ms & 201 \ms & 290\ms & 640\ms & 1.5 \s & 2.5 \s & 3.4 \s & 4.1 \s & 8.8 \s\\
        \hline
        \text{Nemo}  & 45 \ms & 130 \ms & 200 \ms & 290 \ms & 380 \ms & 940 \ms & 2.3 \s & 4.2 \s & 5.9\s & 7.8 \s & 19 \s
    \end{array}
\]

Sage spends additional time in precomputations (as much as 7 minutes in the last example). Using the parameters fixed in~\S\ref{sec:intro},
the total computing time is approximately 10 minutes for each
\(\Walk\), with about 20 seconds spent on each of the largest primes.


\section{Faster $\ell$-isogenies with torsion points}

\begin{algorithm}
    \caption{\(\Step\) using torsion points}
    \label{alg:torsion}
    \KwIn{$E\in \Ell_t(\O)$ and a prime ideal \(\frakell \leftrightarrow
        (\ell,v_\ell)\) such that \(\ell\mid \#E(\FF_{p^d})\)
        for a small \(d\)
    }
    \KwOut{%
        \(E'\) such that \(E \to E'\) is the edge in the isogeny graph
        corresponding to \(\frakell\)
    }
    %
    \Repeat{\(Q \not= 0\) \tcp*[f]{\(Q\) is a point of exact order \(\ell\)}}{
        \(P \gets \textsc{Random}(E(\FF_{p^d}))\)
        \;
        %
        \(Q \gets [C_d/\ell]P\)
        where \(C_d = \#E(\FF_{p^d})\)
        \tcp*{\(C_d\) is a global constant}
        \label{alg:torsion:scalarmult}
    }
    %
    \(S \gets \{[i]Q: 1 \le i \le (\ell-1)/2\}\)
    \tcp*{Use \((\ell-1)/2-1\) repeated additions}
    %
    \Return{\(\textsc{QuotientCurve}(E,S)\)}
    \tcp*{Use Vélu's formulae}
\end{algorithm}


If $E$ has points of order $\ell$
defined over \(\FF_p\), or a small-degree extension of~\(\FF_p\),
then we may replace Algorithm~\ref{alg:step}
with a faster method, Algorithm~\ref{alg:torsion}.
The costly part here is the scalar multiplication in
Line~\ref{alg:torsion:scalarmult}, assuming $\ell$ is not too large;
typically \(C_d/\ell\sim p^d/\ell\).
In order to optimize the number of $\F_p$-operations, we implemented it
using Montgomery curves and \(x\)-line arithmetic~\cite{Montgomery}, 
achieving dramatic performance gains compared to
Algorithm~\ref{alg:step}.
As elliptic curves are not yet supported in Nemo, we implemented our own Montgomery arithmetic. In Sage we use the built-in scalar multiplication. 
Our timings (again on a 3.20GHz Intel Xeon workstation) are as follows:
\[
    \begin{array}{c|c|c|c|c|c|c|c|c}
        d & 1 & 3 & 4 & 5 & 7 & 8 & 9 & 11\\
        \hline
        \text{Sage} & 38 \ms & 290 \ms & 450 \ms & 670 \ms & 1.25 \s & 1.6 \s & 2.0 \s & 3.1  \s \\
        \hline
        \text{Nemo} & 8 \ms & 92 \ms & 150 \ms & 240 \ms & 830 \ms & 1.1 \s & 1.3 \s & 2.2 \s 
    \end{array}
\]
Further experiments show that Julia's compilation and Montgomery arithmetic are equally reponsible for these gains for low degrees, and that Sage handles field extensions more efficiently.

To fully exploit this speed-up, we have to find an \(E_0/\F_p\)
such that many small primes divide \(\#E_0(\FF_{p^d})\) for small~\(d\). 
What can be achieved in a reasonable amount of time, and what speed-up we can obtain, seem interesting questions. 


\begin{thebibliography}{99}

\bibitem{BMSS}
A. Bostan, F. Morain, B. Salvy, \'E. Schost.
\newblock Fast algorithms for computing isogenies between elliptic curves.
\newblock {\em Math.~Comput.} 77:263:1755--1778 (2008).

\bibitem{Couveignes}
J.-M. Couveignes.
\newblock Hard homogeneous spaces.
\newblock Preprint (2006).

\bibitem{Nemo}
C. Fieker, W. Hart, T. Hofmann, F. Johansson et. al.
\newblock the Nemo computer algebra package, version 0.6.0, \url{http://nemocas.org}.

\bibitem{GMP}
T. Granlund and the GMP development team.
\newblock The GNU Multi Precision Arithmetic Library, \url{http://gmplib.org/}

\bibitem{Montgomery}
P. L. Montgomery.
\newblock Speeding the Pollard and Elliptic Curve Methods of Factorization.
\newblock {\em Math.~Comput.} 47:177:243--264 (1987).

\bibitem{Pari}
The PARI~Group.
\newblock PARI/GP version {\tt 2.9.0}, Univ. Bordeaux, 2016, \url{http://pari.math.u-bordeaux.fr/}.

\bibitem{RS}
A. Rostovtsev, A. Stolbunov.
\newblock Public-key cryptosystem based on isogenies.

\bibitem{Sage}
The Sage Developers.
\newblock {S}ageMath, the {S}age {M}athematics {S}oftware {S}ystem ({V}ersion
  7.5.1),
\newblock 2017, \url{http://www.sagemath.org}

\bibitem{NTL}
V. Shoup.
\newblock NTL : a library for doing number theory.
\newblock \url{http://www.shoup.net/ntl/}

%\bibitem{Waterhouse}
%W. C. Waterhouse.
%\newblock Abelian varieties over finite fields.
%\newblock {\em Ann. sci. \'E.N.S.} 2:4:521--560, 1969.

%\bibitem{Attack}
%S. Galbraith.
%\newblock Constructing isogenies between elliptic curves over finite fields.
%\newblock {\em LMS J. Comput. Math.} 2:118--138, 1999.

%\bibitem{Kohel}
%D. Kohel.
%\newblock Endomorphism rings of elliptic curves over finite fields.

%\bibitem{Flint}
%W. Hart, F. Johansson, S. Pancratz.
%\newblock {FLINT}: {F}ast {L}ibrary for {N}umber {T}heory, 2013,
%\newblock Version 2.4.0, \url{http://flintlib.org}

\end{thebibliography}

\end{document}
